\documentclass[10pt,pdf,hyperref={unicode}]{beamer}

\mode<presentation>
{
\usetheme{boxes}
\beamertemplatenavigationsymbolsempty

\setbeamertemplate{footline}[page number]
\setbeamersize{text margin left=0.5em, text margin right=0.5em}
}

\usepackage[utf8]{inputenc}
\usepackage[english, russian]{babel}
\usepackage{bm}
\usepackage{multirow}
\usepackage{ragged2e}
\usepackage{indentfirst}
\usepackage{multicol}
\usepackage{subfig}
\usepackage{amsmath,amssymb}
\usepackage{enumerate}
\usepackage{mathtools}
\usepackage{comment}
\usepackage{multicol}

\usepackage[all]{xy}

\usepackage{tikz}
\usetikzlibrary{positioning,arrows}

\tikzstyle{name} = [parameters]
\definecolor{name}{rgb}{0.5,0.5,0.5}

\usepackage{caption}
\captionsetup{skip=0pt,belowskip=0pt}

\newtheorem{rustheorem}{Теорема}
\newtheorem{russtatement}{Утверждение}
\newtheorem{rusdefinition}{Определение}

% colors
\definecolor{darkgreen}{rgb}{0.0, 0.2, 0.13}
\definecolor{darkcyan}{rgb}{0.0, 0.55, 0.55}

\AtBeginEnvironment{figure}{\setcounter{subfigure}{0}}

\captionsetup[subfloat]{labelformat=empty}

%----------------------------------------------------------------------------------------------------------

\title[Вводная лекция]{Создание Интеллектуальных Систем \\ Код Вычислительного Эксперимента}
\author{А.\,В.\,Грабовой, О.\,Ю.\,Бахтеев}

\institute[]{Московский физико-технический институт}
% \date[2022]{\small 10\;февраля\;2022\,г.}

%---------------------------------------------------------------------------------------------------------
\begin{document}

\begin{frame}
\titlepage
\end{frame}

%----------------------------------------------------------------------------------------------------------
\section{Требования}
\begin{frame}{Требования}
\bigskip

\begin{enumerate}
    \item Существование проекта согласно шаблона$^*$.
    \item Полнота документации кода.
    \item Наглядность работы модели.
    \item Код должен быть структурирован.
    \item Проверка корректности кода.
    \item Результаты эксперимента должны быть воспреизведимы.
    \item Оформление кода.
\end{enumerate}

\end{frame}

%----------------------------------------------------------------------------------------------------------
\section{Существование проекта согласно шаблона}
\begin{frame}{Существование проекта согласно шаблона}
\bigskip
\begin{enumerate}
    \item Репозиторий должен быть создан согласно шаблону. Все заглушечные функции и readme.md должны быть адаптированы под проект. Шаблон репозитория по ссылке: \url{https://github.com/Intelligent-Systems-Phystech/ProjectTemplate}.
    \item Информация о репозитории должна быть внесена в файл с проектами: \url{https://github.com/Intelligent-Systems-Phystech/.github/blob/master/profile/repository_structure_rtfm.md}.
\end{enumerate}
\end{frame}

%----------------------------------------------------------------------------------------------------------
\section{Полнота документации кода}
\begin{frame}{Полнота документации кода}
\bigskip
\begin{enumerate}
    \item Все функции и классы кода прокомментированы достаточно полно, прописано описание параметров, выхода.
    \item Документация, а также *md-файлы должны быть хорошо и аккуратно оформлены.
    \item Настроена генерация sphinx документации и размещение в github pages (из шаблона).
\end{enumerate}
\end{frame}
%----------------------------------------------------------------------------------------------------------
\section{Наглядность результатов}
\begin{frame}{Наглядность результатов}
\bigskip
\begin{enumerate}
    \item В репозитории находится ноутбук-демонстрация основных результатов, максимально наглядно и иллюстративно описывающий проделанную работу.
    \item В readme должны содержаться ссылки на демо, а также шаги для его запуска.
\end{enumerate}
\end{frame}
%----------------------------------------------------------------------------------------------------------
\section{Структурирование кода}
\begin{frame}{Структурирование кода}
\bigskip
\begin{enumerate}
    \item Код логически разнесен по модулям и пакетам.
    \item Основной код работы должен быть вынесен за пределы ноутбуков, демонстраций и пр.
\end{enumerate}
\end{frame}
%----------------------------------------------------------------------------------------------------------
\section{Корректность кода}
\begin{frame}{Корректность кода}
\bigskip
\begin{enumerate}
    \item Для основного кода работы написаны тесты, проверяющие корректность работы функцией и классов.
\end{enumerate}
\end{frame}
%----------------------------------------------------------------------------------------------------------
\section{Воспроизведимость результатов}
\begin{frame}{Воспроизведимость результатов}
\bigskip
\begin{enumerate}
    \item В репозитории явно указаны шаги для запуска экспериментов и демонстраций, с указанием ОС, установленных пакетов и их версий.
    \item Для python-проектов плюсом будет наличие файла requirements.txt.
\end{enumerate}
\end{frame}
%----------------------------------------------------------------------------------------------------------
\section{Оформление кода}
\begin{frame}{Оформление кода}
\bigskip
\begin{enumerate}
    \item Код оформлен в соответствии с соглашениями, принятыми в языке программирования (PEP-8).
    \item В случае отсутствия явных соглашений, требуется руководствоваться принципом разумности.
\end{enumerate}
\end{frame}

\end{document}